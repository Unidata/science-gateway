\documentclass[conference]{IEEEtran}
\IEEEoverridecommandlockouts
% The preceding line is only needed to identify funding in the first footnote. If that is unneeded, please comment it out.
\usepackage{cite}
\usepackage{url}
\usepackage{amsmath,amssymb,amsfonts}
\usepackage{algorithmic}
\usepackage{graphicx}
\usepackage{textcomp}
\usepackage{xcolor}
\def\BibTeX{{\rm B\kern-.05em{\sc i\kern-.025em b}\kern-.08em
    T\kern-.1667em\lower.7ex\hbox{E}\kern-.125emX}}
\begin{document}

\title{Unidata Science Gateway Reimagined: Unifying Access to Educational and Research Resources \\
  \thanks{\small Presented at Gateways 2022, San Diego, USA, October 18--20, 2022.
    \protect\url{https://zenodo.org/communities/gateways2022/}}
}

\author{\IEEEauthorblockN{Julien Chastang}
\IEEEauthorblockA{\textit{UCAR-Unidata} \\
Boulder, USA \\
chastang@ucar.edu} \\
\and
\IEEEauthorblockN{Nicole~Corbin}
\IEEEauthorblockA{\textit{UCAR-Unidata} \\
Boulder, USA \\
ncorbin@ucar.edu} \\
\and
\IEEEauthorblockN{Ethan~Davis}
\IEEEauthorblockA{\textit{UCAR-Unidata} \\
Boulder, USA \\
edavis@ucar.edu} \\
\and
\IEEEauthorblockN{Bobby~Espinoza}
\IEEEauthorblockA{\textit{UCAR-Unidata} \\
Boulder, USA \\
respinoza@ucar.edu} \\
\and
\IEEEauthorblockN{Tanya~Vance}
\IEEEauthorblockA{\textit{UCAR-Unidata} \\
Boulder, USA \\
tavance@ucar.edu} \\
}

\onecolumn

\maketitle

\begin{abstract}
Unidata is a diverse community of education and research institutions with the common goal of sharing geoscience data and the tools to access and visualize that data. For more than 30 years, the Unidata Program Center has been providing data, software tools, and support to enhance Earth-system education and research. Funded primarily by the National Science Foundation (NSF), Unidata is one of the University Corporation for Atmospheric Research (UCAR)'s Community Programs (UCP). The Unidata Science Gateway is a virtual space where the academic and research communities can collaborate, share resources, and learn from each other in a classroom or workshop setting. The Gateway combines Unidata technologies with the NSF Jetstream2 cloud computing platform and open-source software such as Project Jupyter to create an environment where community members can access, analyze, and visualize real-time and case study Earth system science data easily and efficiently.
\\
The NSF-funded Jetstream cloud-computing platform allows researchers and educators access to computational infrastructure and a collection of virtual machines with various capabilities including GPUs and large capacity instances. Since 2015, Unidata has been providing persistent geoscientific data services and software tools targeted at atmospheric science initially on Jetstream and now the newly launched Jetstream2 cloud. The array of Unidata and open source technologies running on the Unidata Jetstream allocation is branded under the concept of the Unidata Science Gateway.
\\
One of the goals of the Unidata Science Gateway is to democratize access to powerful scientific computing systems for education and research. Through the Science Gateway, Unidata has provided resources to help hundreds of students access pre-configured scientific computing environments allowing them to focus on their learning rather than on managing software. We have also hosted large datasets to enable data-proximate computing for student projects, giving access to computational infrastructure and expertise that may not be available at students’ institutions.
\\
While the Unidata Science Gateway has been successful in attaining our objectives, we would like to improve and expand our web presence, building a portal that allows users to more easily access educational, computing, and data resources. In particular, we aim to revamp our current gateway interface to become a more dynamic hub for learning, data, and research. This poster will outline our vision; we hope to obtain early feedback and community involvement from key stakeholders and community members to inform our choices as we improve the Unidata Science Gateway.
\end{abstract}


\section*{Acknowledgment}

For Jetstream2 and JupyterHub expertise, we thank Andrea Zonca (San Diego Supercomputing Center), Jeremy Fischer, Mike Lowe (Indiana University), the NSF Jetstream2 (\url{https://doi.org/10.1145/3437359.3465565}) team, and the NSF XSEDE Extended Collaborative Support Service (ECSS) program (\url{https://doi.org/10.1007/978-3-319-32243-8_1}).


\end{document}
