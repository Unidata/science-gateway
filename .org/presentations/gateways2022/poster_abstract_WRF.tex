\documentclass[conference]{IEEEtran}
\IEEEoverridecommandlockouts
% The preceding line is only needed to identify funding in the first footnote. If that is unneeded, please comment it out.
\usepackage{cite}
\usepackage{amsmath,amssymb,amsfonts}
\usepackage{algorithmic}
\usepackage{graphicx}
\usepackage{textcomp}
\usepackage{xcolor}
\usepackage{url}
\def\BibTeX{{\rm B\kern-.05em{\sc i\kern-.025em b}\kern-.08em
    T\kern-.1667em\lower.7ex\hbox{E}\kern-.125emX}}

%% https://tex.stackexchange.com/questions/474382/ieee-styling-authors-with-5-columns
%% For inserting a ``newline'' in the author's list
\makeatletter
\newcommand{\newlineauthors}{%
    \end{@IEEEauthorhalign}\hfill\mbox{}\par
    \mbox{}\hfill\begin{@IEEEauthorhalign}
}
\makeatother

\begin{document}

\title{Democratizing Access to Atmospheric Modeling with WRF employing NSF Cloud Computing
Resources \\
\thanks{\small Presented at Gateways 2022, San Diego, USA, October 18--20, 2022.
\protect\url{https://zenodo.org/communities/gateways2022/}}
}

\author{\IEEEauthorblockN{Bobby Espinoza}
\IEEEauthorblockA{\textit{UCAR -- Unidata} \\
Boulder, CO USA \\
respinoza@ucar.edu}
\and
\IEEEauthorblockN{Julien Chastang}
\IEEEauthorblockA{\textit{UCAR -- Unidata} \\
Boulder, CO USA \\
chastang@ucar.edu}
\and
\IEEEauthorblockN{Jeff Weber}
\IEEEauthorblockA{\textit{UCAR -- Unidata} \\
Boulder, CO USA \\
jweber@ucar.edu}
\newlineauthors
\IEEEauthorblockN{Dennis Dye}
\IEEEauthorblockA{\textit{Southwestern Indian Polytechnic Institute} \\
Alburquerque, NM USA \\
dennis.dye@bie.edu}
\and
\IEEEauthorblockN{Peter Romine}
\IEEEauthorblockA{\textit{Navajo Technical University} \\
Crownpoint, NM USA \\
promine@navajotech.edu}
}

\onecolumn

\maketitle

\begin{abstract}

    The Unidata Program Center, one of the University Corporation for
    Atmospheric Research’s (UCAR’s) Community Programs (UCP), is devoted to
    providing software, data, and other resources to the Earth science community
    for both education and research. Primarily funded by the National Science
    Foundation (NSF), Unidata has been allocated resources on the NSF’s
    Jetstream2 (JS2) cloud computing platform. The Unidata Science Gateway is a
    collection of Unidata technologies hosted on JS2, including software for
    sharing and distributing data such as THREDDS Data Servers, AWIPS
    Environmental Data EXchange (EDEX) servers, and Internet Data Distribution
    (IDD) network nodes, deployed with and alongside technology such as Docker,
    Kubernetes, and Project Jupyter software.
    \\

    One of the main advantages of a cloud computing platform is the ability to
    provide equitable access to powerful hardware components. For example, The
    Unidata Science Gateway benefitted hundreds of students by providing access
    to preconfigured JupyterHub servers in a classroom setting. Unidata’s
    provision of remote computing environments ensures that educators and
    students spend less time configuring environments/software and more time
    focusing on education.
    \\

    This philosophy of equitable access to resources extends beyond education.
    The Unidata Program Center is involved with the Southwestern Indian
    Polytechnic Institute and Navajo Technical University under NSF grant
    21-533. This grant is awarded to advance data sovereignty by creating a
    sovereign network and providing capacity building for environmental
    monitoring for Tribal Nations.
    \\

    Computational and bandwidth constraints make running the Weather Research
    and Forecasting Model (WRF) in a cloud environment more attractive than
    using local computing hardware.  Additionally, deploying a containerized
    version of WRF through Docker removes the legwork of compiling WRF and all
    of its dependencies, and allows for reproducibility and distribution of the
    WRF configuration, if desired \cite{hacker-BAMS}, \cite{c-nwp}. Using
    Unidata Science Gateway resources, a high-resolution WRF model will be run
    over the Navajo Nation, and the model output will be placed on a co-located
    Unidata RAMADDA server. The science gateway’s RAMADDA server will push
    relevant parameters of the model output directly to RAMADDA servers at SIPI
    and NTU for local analysis, while storing the entire output in the cloud. At
    SIPI and NTU, real time IDV bundles will be used to display and analyze the
    WRF model output, and create products for students and for the community.

\end{abstract}

\section*{Acknowledgments}

For JetStream2 and JupyterHub expertise, we thank Andrea Zonca (San Diego
Supercomputing Center), Jeremy Fischer, Mike Lowe (Indiana University), the NSF
Jetstream2 (\url{https://doi.org/10.1145/3437359.3465565}) team, and the NSF XSEDE
Extended Collaborative Support Service (ECSS) program
(\url{https://doi.org/10.1007/978-3-319-32243-8_1}).
\\

This collaboration was made possible by NSF Grant 21-533: CISE-MSI Program.

\begin{thebibliography}{00}
\bibitem{hacker-BAMS} G. Eason, B. Noble, and I. N. Sneddon, ``A Containerized
    Mesoscale Model and Analysis Toolkit to Accelerate Classroom Learning,
    Collaborative Research, and Uncertainty Quantification'' Bulletin of the
    American Meteorological Society, 98(6), 2017.
\bibitem{c-nwp} https://github.com/NCAR/container-dtc-nwp
\end{thebibliography}

\end{document}
